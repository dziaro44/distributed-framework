\documentclass{article}

\usepackage[english]{babel}
\usepackage{filecontents}
\usepackage[letterpaper,top=2cm,bottom=2cm,left=3cm,right=3cm,marginparwidth=1.75cm]{geometry}

\usepackage{amsmath}
\usepackage{csquotes}
\usepackage{graphicx}
\usepackage{amsthm}
\usepackage{amsfonts}

\usepackage[colorlinks=true, allcolors=blue]{hyperref}

\usepackage{biblatex}
\begin{filecontents*}[overwrite]{general.bib}
 @article{fokkink2005simplifying,
  title={On an improved algorithm for decentralized extrema finding in circular configurations of processors},
  author={Randolph Franklin},
  journal={Communications of the ACM},
  volume={25},
  number={5},
  pages={336--337},
  year={1982},
  publisher={ACM}
}
\end{filecontents*}
\addbibresource{general.bib}
\nocite{*}

\usepackage{indentfirst}

\usepackage{cleveref}
\newtheorem{lemma}{Lemma}
\newtheorem{theorem}{Theorem}
\begin{document}
\section*{Franklin's leader election algorithm for bidirectional rings}

\subsection*{Introduction}
Franklin's algorithm discussed here performs leader election in bidirectional rings with asynchronous communication. Each process (node) has an unique identifier $id$ known to itself.
The pessimistic number of messages is at most $O(nlog_2 n)$.

\subsection*{The algorithm - description}
At any time of the algorithm's execution, each process is either active, passive or a leader. There are either no leaders or exactly one leader at any given moment. Before an overview, let's define 'active neighbours'. By 'left/right active neighbour of node $a$' we mean the closest active node when traversing the ring in the direction specified by $a$'s left/right link. Nodes may have different understanding of which side is 'left' and which 'right'; this is not an issue. 

The algorithm is split into rounds. In each round an active process $a$ sends two messages with its $id$ to left and right active neighbours and receives two messages with $idLeft$ and $idRight$ from these nodes. If either of received identifiers is greater than $a$'s $id$, then it becomes passive. If $idLeft==id$ or $idRight==id$, $a$ becomes the leader. If neither of above conditions is met, $a$ remains active.

Passive node $a$ simply forwards messages in from either direction in the same direction. If a message is $final$, then $a$ sets message's value as leader's identifier and terminates. It will become clear that an active node cannot receive a $final$ message. A leader node only sends a single $final$ message and terminates when it comes back (forwarded by other nodes, all passive). 

\subsection*{The algorithm - pseudocode}
\subsubsection*{Process $a$ maintains:}
\begin{itemize}
    \item $id_a$ --- a unique identifier
    \item $leaderId$ --- leader's $id$, must be set before termination
    \item $state_a \in \{active,\ passive,\ leader\}$
\end{itemize}

\subsubsection*{Message format:}
\begin{itemize}
    \item $id$ --- original sender's id
    \item $isFinal$ --- true for the single message sent by leader and then forwarded; false otherwise
\end{itemize}

\subsubsection*{Procedures}
\textbf{Initialization:} all processes are active and know their $id$'s

\textbf{Passive process:} Upon receipt of a message $msg$ , passes on the message. If $msg.isFinal$ is true, sets $leaderId$ to $msg.id$ and terminates.

\textbf{Active process:} In each phase $a$:
\begin{itemize}
    \item sends message with $id_a$ to its left and right neighbours,
    \item receives messages with $id_{left}$ and $id_{right}$ from its active neighbours;
    \item if $id_{left}>id_a$ or $id_{right}>id_a$, then $a$ becomes passive,
    \item if $id_{left}==id_a$ or $id_{right}==id_a$, then $a$ becomes a leader
\end{itemize}

\textbf{Leader process:} Leader process $a$ sets $leaderId = i_a$ and sends message with $id_a$ and $isFinal$ set to $true$ to its left or right neighbour. When it receives the message back, terminates.

\clearpage

\subsection*{Correctness}
Correctness of Franklin's algorithm  stems from the following observations:
\begin{itemize}
    \item If there is a single $leader$ node and other nodes are passive, all nodes will terminate with correct $leaderId$ set. This is clear from the protocol.
    \item If there is a single $leader$ node, all other nodes are passive. This is because in order to become a leader, node $a$ has to receive a message with its $id$. Such message can only be sent by $a$ (ad identifiers are unique). It can be received back only when it traversed entire ring, and this implies that all nodes other than $a$ are passive.
    \item There can't be two leader nodes. This fact follows from the above point.
    \item If there are at least two active processes in a given round, at least one of them must become passive at the end of this round. If this was not true, each of the active nodes would have $id$ bigger than its neighbour identifiers, which is impossible.
    \item The algorithm is deadlock-free. This is clear from the protocol. 
\end{itemize}

\subsection*{Complexity analysis}
Let $n$ be the size of the ring. In each round, at least half of the active nodes become passive, as in order to remain active a node must have $id$ greater than its active neighbours identifiers. Therefore, after at most $FLOOR(log_2 n)$ steps there is only one active process. When only one process $a$ is active, it will send two messages which will be forwarded by the other nodes and come back to $a$ - $2n$ message passes. Then $a$ becomes leader and after the next $n$ passes of the final message, the algorithm terminates. As there were exactly $2n$ message passes in any round with $k \ge 2$ active nodes, we conclude that the time complexity is:
$$
2n*FLOOR(log_2 n) + 3n \in O\left(n log_2 n\right)
$$
It is stated in $[1]$ that average number of messages passed was $O\left(n\right)$  in sample tests.

\printbibliography

\end{document}
